\documentclass[a4paper,english, 12pt, twoside]{article}
\usepackage[utf8]{inputenc}
\usepackage[T1]{fontenc}
\usepackage[english]{babel}
\usepackage{epsfig}
\usepackage{graphicx}
\usepackage{caption}
\usepackage{subcaption}
\usepackage{amsfonts, amssymb, amsmath}
\usepackage{listings}
\usepackage{float}
\usepackage[top=2cm, bottom=2cm, left=2cm, right=2cm]{geometry}

\usepackage[]{biblatex}
\bibliography{bibtex_ref_test.bib}

\renewcommand{\d}{\partial}
% \renewcommand{\dell}{\nabla}

%opening
\title{Diffusion Processes in the extracellular space of the brain}
\author{Fredrik E Pettersen\\ f.e.pettersen@fys.uio.no}
\begin{document}

\maketitle
% \titlepage

\begin{abstract}
This is a project in computational neuroscience based on some 8 articles describing various aspects of diffusion in the extracellular space of the brain. 
\end{abstract}
\tableofcontents
\newpage
\section{Background material}
For all practical purposes, the brain of a rat is considered equal to the brain of a human in this project.
\subsection{Basics of the brain}

The human brain consists of two types of cells; the neurons and neuroglia. 
Neurons are tasked with signal processing and transport, while the glia are thought to have more janitoral tasks. 
The neurons are bathed in a salt solution that is mainly $Na^+$ and $Cl^-$. 
Inside the neurons, a highly regulated salt solution of mainly $K^+$ sets up a potential difference relative to the outside of the cell of around $-65$mV.
The neurons are in constant communication with each other through action potentials, which are disturbances in the membrane potentials on neurons. 
These action potentials are generated in the body of the cell, called the soma, and then propagate down the axon without loss of amplitude. 
This is achieved by constantly amplifying the signal using ion pumps (see the Hodgin-Huxley model of the action potential).
After propagating down the axon, the action potential reaches a synapse which is a gate to another neuron. 
If the action potential is of significant strength, vesicles carrying neurotransmitters merge with the synapse membrane, letting the neurotransmitters diffuse to the dendrite of the other neuron. 
If enough neurotransmitters reach the post-synaptic side, the signal continues propagating to the soma of this neuron, and the entire process starts over again. \\
The interest of this project lies, mainly, in the diffusion processes that take place in the space between these types on cells, the so-called extracellular space (ECS). 
This is a narrow space ($\sim 10-100$ nm \cite{nicholson2001diffusion}) with a highly complicated geometry (figure \ref{ECS}). 
Surprisingly, the ECS adds up to a total of $20\%$ of the total brain volume. 
We can understand this by realizing that every part of a cell must be separated from another cell by the ECS. 
Since the cells consists of axons and dendrites which can be viewed as (somewhat) fractal, we see that this eventually means separating a vast ammount of surface area from other surface areas.

\begin{figure}[H]
 \centering
 \includegraphics[scale=0.8]{brain.jpg}
 \caption{Electron micrograhp of a small region of the cerebral cortex of a rat with a prominent synapse. 
 The black areas os the picture indicate the ECS, which may be reduced in size as a conequence of the processing. 
 The asterix (*) indicates the postsynaptic side of a synapse on a dendrite. 
 On the other side of the synaptic cleft one can make out the pre-synaptic terminal containing several small, round vesicles filled with neurotransmitter molecules. 
%  The scale bar under the figure indicates a distance of about 1$\mu$m. 
 Figure taken from Nicholson \cite{nicholson2001diffusion}.}
 \label{ECS}
\end{figure}

The ECS is thought to support the diffusion of oxygen and nutrients to the neurons and glia, and diffusion of carbon dioxide and other waste from these cells through the blood - brain barrier and into the bloodflow. 

\subsection{Diffusion in general}
Diffusion is a transport process which in it's well known macroscopic form has been attributed to Adolf Fick. 
In 1855, building on the earlier experimental work of Graham, Fick formulated the macroscopic law later known as Fick's law
\begin{equation}
 \text{flux} = -D\times\text{concentration gradient}
\end{equation}
Where $[D] = \frac{m^2}{s}$ is the diffusion constant. Fick's law leads to the well known partial differential equation in the concentration.
\begin{equation}\label{diffusion_eq}
 \frac{\d C}{\d t} = D\nabla^2C
\end{equation}

Einstein later (1905) proposed the most usefull relation between the diffusion constant and a fluid viscosity 
\begin{equation}\label{einstein_viscosity}
D = \frac{k_B T}{6\pi \eta r}
\end{equation}

.
There are several relations on this form, relating the diffusion constant to various other easily measurable quantities. 
The most relevant for the study of diffusion in the ECS would be
\begin{equation}\label{einstein}
 \langle r^2\rangle = 2dDt
\end{equation}
in the limit of large t (that is in steady state). Equation \ref{einstein} relates the root-mean-square displacement of a particle after a time t (t is large) in a d dimensional space. 
There are two good reasons to consider the transport mechanisms in the ECS as diffusion processes. 
First of all, diffusion goes seamlessly from micro to macro scale. In our case, this is perfect since the channels of the ECS are very narrow, but not nececarily narrow enough to consider all the microscopical effects. Second, the geometry of the problem is somewhat similar to diffusion in porous media.\\
TO DO!!!

\subsection{Why diffusion in ECS}
Though there are several reasons to study diffusion processes in the ECS this project has a specific goal in mind. 
The einstein relation \ref{einstein_viscosity} relates the diffusion constant to the viscosity of the medium in which the diffusion is taking place. 
From the definition of viscosity, $\mu$, we have 
\begin{equation}
v_d = \mu F 
\end{equation}
where $F = qE$ is the standard electical force acting on a charged particle. 
We can also define the current from the drift velocity of the particles as 
\begin{equation}
j = cqv_d = \sigma E 
\end{equation}
where $\sigma = c\mu q^2$ is the electrical conductance, in this case, of the ECS. 
Inserting this in the einstein relation \ref{einstein_viscosity} lets us express the conductivity in terms of the diffusion constant 
\begin{equation}
\sigma = \frac{cq}{k_B T}D 
\end{equation}

We are interrested in the extracellular conductance for measurement purposes. \\
TO DO!! SAY SOMETHING ABOUT NETWORK MODELS?

\section{Mathematical models}
Diffusion in the ECS is, naturally, modelled by a diffusion equation, but rather a modified one than the basic diffusion equation \ref{diffusion_eq}. 
Since the geometry of the ECS is very narrow, molecules diffusing in this space will not be subject to free diffusion, as the normal diffusion equation assumes. 
The diffusing molecules will bounce off cell membranes (we are now considering macromolecules on such a scale that speaking off a cell membrane makes sense) and other molecules. 
There may even be molecules absorbed by cells, getting stuck onto membranes or going through similar processes. 
We therefore see it fit to introduce a modified version of the diffusion equation which is more similar to the diffusion equation governing diffusion in porous media. 
\begin{equation}\label{modified_diffusion}
  \frac{\d C}{\d t} = D^*\nabla^2C +\frac{s}{\alpha} -k'C
\end{equation}
Where we have introduced an effective diffusion constant $D^*$ defined from the tortuosity, $\lambda = \sqrt{\frac{D}{D^*}}$, which is a parameter saying something about the reduction in the diffusion constant compared to free diffusion (usually measured in a low percentage agar solution in water). 
The tortuosity can also be interpreted as a measure of the ``twistyness'' of the media in which the diffusion is taking place. 
$\alpha$ is defined as the relative volume fraction the ECS accounts for, s is a source term, and the $k'C$ term models the uptake of the diffusing molecules by cell membranes etc. 
Note that equation \ref{modified_diffusion} is only one of several possible equations used to model this kind of diffusion, and that terms accounting for the (now assumed absent \cite{nicholson2001diffusion}) bulk flow in the ECS, and other possibly contributing terms, are not included. 
The model does, however, illustrate the general idea behind the modelling of diffusion in the ECS.

\subsection{Numerical simulations of tortuosity}
In 2003 Hrab\v{e}tov\'{a} and Nicholson conducted numerical simulations of diffusion in an artificial, porous 3D media to determine the tortuosity, $\lambda$, as a function of the volume fraction, $\alpha$ \cite{hrabetova2004contribution}. 
The ECS was modeled as the gaps between geometric figures, starting with cubes and advancing to more complicated shapes (see figure \ref{geometries}). 
The models all have a regular spacing in common. \\
In their simulations Hrab\v{e}tov\'{a} and Nicholson found that the tortuosity reached a maximum of $\lambda_g(\alpha=0) = \sqrt{3/2}\approx 1.225$, which is substantially different from the normal value of $\lambda = 1.6$. 
These results suggest that other effects than geometric hindrance must be considered in the explanation of the effective diffusion constant. 
Personally I suspect that the observed maximum tortuosity comes from the regular geometry used in the simulations, and I have tested this hypothesis by the means of molecular dynamics (see section \ref{MD}). 


\section{Measurement techniques for brain diffusion characteristics}
Measurement of the diffusion constant in the ECS can be done in 4 different ways, where one has the obvious advantage of not having to remove the brain from the scull. 
For in vitro measurements there main types of measurement are optical, and ionsensitive microprobes. 

\subsection{Optical measurements}
In 1993 Nicholson and Tao developed a new method, the integrative optical imaging (IOI) method, to measure diffusion properties in the ECS \cite{nicholson1993hindered} \cite{nicholson2001diffusion}. 
The basic approach is to eject a small volume ($\sim 1$nL or less) of macromolecules labelled with a fluorescent tag into the tissue of interest (here the ECS), and record the spatial distribution of these molecules by a cooled charge-coupled device (CCD) camera every few seconds. 
This method also (naturally) requires the use of a microscope. 
The CCD camera transfers the pictures to a computer which calculates the effective diffusion constant. 
A comparison with unhindered diffusion is required as it is for the TMA measurments also. \\
The IOI method has proved to work rather well, but it has one fundamental problem to overcome; the camera only takes pictures in 2d, and hence the slices used for these measurements must be optically thin. 
A further complication of this is that the diffusion should happen in 3d, but will be limited. 
On a brighter note, the uptake term in equation \ref{modified_diffusion} is assumed to be zero ($k' = 0$) since the molecules are so large.

\subsection{TMA$^+$ measurements}
All TMA measurements are usually done on brain slices kept under well controlled conditions. 
Though the method is invasive, it does not have to be done in vitro, but an in vivo measurement requires anesthetized animals and a lot of skill!\\
Tetramethylammonium (TMA$^+$) is the simplest quaternary ammonium cation consisting of four methyl groups attached to a central nitrogen atom, and is positively charged. 
Measurements using TMA$^+$ rely on the controlled release of very small ammounts of TMA from a micropipette (using for example pressure ejection) into the tissue of interest, and measuring the corresponding concentration change some distance away (usually $\sim100\mu$m). 
The change in concentration is measured by an appropriate ion-selective microelectrode. 
Using ion-selective microelectrodes we can mesure the TMA concentration in time as well as space with the real time iontophoresis (RTI) method developed by Nicholson and Phillips in 1981 \cite{} and get a resolution of the order of a minute.\\
Within the realm of TMA measurements there are a few varieties. 
As suggested, we can measure both the spatial concentration distribution, making an analogous method to the radiotracer method. 
One can also utilize the RTI methos to get a time resolution as well as a spatial one. 
A third possible method was developed by Chen and Nicholson in 2002 \cite{chen2002measurement} using a sinusoidal source. 
Because the ions have to diffuse to the probe we will measure a phase lag in the oscillating steady state solution at the microelectrode. 

\subsection{Radiotracer methods}
The radiotracer methods are perhaps the most intuitive methods for measuring diffusion characteristics in the ECS, and were also the first methods applied quantitatively (1962 \cite{}). 
Although the method is not generally in use today because it requires interaction with radioactive substances (which one now tend to limit), the results of these early experiments are still used as verification because they offer an independent reference and the method is still sound. 
The basic method is to perfuse the bilateral ventriculoscisternal cavities of an anesthetized animal, usually a dog or monkey, with a radiolabeled probe molecule for a longer period. Often several hours. 
After this period was over, the brain of the animal was quickly removed, frozen and sliced at $0.4-1$ mm thickness parallel to the perfusion. 
One could now measure the spatial distribution of radioactivity in each slice to determine the diffusion parameters $D^*$ and $\alpha$. 
As mentioned, modern health and safety measures limits the interaction with radioactive substances. 
It is also more and more frowned upon to use larger animals, like monkeys, in these kinds of experiments, and the TMA$^+$ measurements are just as good or better. 
The radiotracer methods are therefore not commonly used today.

\subsection{Diffusion Tensor Imaging}
This is a non-invasive measurement which has its obvious advantages in it's ability to be used on living humans. 
Diffusion tensor magnetic resonance imaging (DTI) is, as the name suggests, a type of magnetic resonance imaging. 
In isotropic media, the diffusion process is fully described by a single scalar coefficient, the diffusion constant. 
With isotropy we mean that the medium in which the diffusion process is taking place is uniform in all directions which means that the diffusion will be equal in all directions. 
The ECS, however (and tissue in general) is anisotropic. 
In the presence of anisotropy, diffusion can no longer be characterized by a single scalar constant, but requires a tensor $\mathbf{D}$, which fully describes the mobility along each direction and the correlation between these directions \cite{le2001diffusion}. 
The diffusion tensor is given in \ref{diffuion_tensor}. We notice that it must be symmetric, meaning that we have $D_{ij} = D_{ji}$.
\begin{align}\label{diffuion_tensor}
 \mathbf{D} = \left(\begin{array}{ccc}
                     D_{xx} & D_{xy} & D_{xz}\\
                     D_{yx} & D_{yy} & D_{yz}\\
                     D_{zx} & D_{zy} & D_{zz}
                    \end{array}\right)
\end{align}

DTI measures the self diffusion tensor of water molecules in the ECS using six or more gradients. 

\section{Other possible modeling approaches}
In this part we will briefly look to other fields of study which might have similar problems to solve. 
The main issues of the established models for simulating diffusion in the ECS is the length scale on which it happens. 
The einstein relation is (to my knowledge) only derived in homogenous media, which means that the mean free path for a diffusing molecule must be much larger than the size of the molecule. 
It the ECS we are typically in a grey are with respect to the validity of the homogenous media assumption. \\
There are, however, some fields which operate with similar geometries within numerical statistical mechanics, and we will look closer at two possibly relevant fields here.

\subsection{Random walks/flow on a percolating cluster}
Percolation theory is the study of connectivity in random media. There are ways to study percolation, in this project we will only look briefly at the basics. The theory will be described in 2d, but it is trivially generalized to 3 or more dimensions.\\
We typically start with a matrix of random numbers drawn from the uniform distribution, and pick some porosity, $p\in[0,1]$. 
For each site (entry) $e_{ij}$ in the matrix we will now say that it is occupied if $e_{ij} \leq p$ and that two occupied sites are connected if they are neighbours (depending on what kind of neighbouring criteria we want). 
This gives us clusters of connected, occupied sites with complex geometries \ref{perc}. 
If a cluster stretches from one side of the matrix to the other we say that we have percolation, and call that cluster the percolating cluster. 
Now, there are many interesting aspects of this kind of system in itself, but we would like to model diffusion in the ECS with this kind of system. 
To do this we can either simulate flow on the percolating cluster (which might not be a good model considering the absence of bulk flow in the ECS), or we can do Monte Carlo simulations of random walks on the percolating cluster. 
For a random walker we can easily measure the diffusion constant using equation \ref{einstein} where $\langle r^2\rangle$ denotes the rms displacement of the walker relative to its starting position. \\
Interestingly, percolation theory might offer an explanation to some of the observations done by Chen and Nicholson in 2000 (cited in section 5 of \cite{hrabetova2004contribution}) with respect to unexpected behaviour of $\lambda$ under hyperosmotic stress. 
This stress increases $\alpha$ from $0.24$ to $0.42$, but $\lambda$ remains more or less constant. 
Percolation theory tells us that as the porosity (analogous to $\alpha$) increases, this will increase the mass (size) of the percolating cluster, but the contribution will be to the so-called dangling ends. These are the parts of the percolating cluster which, if removed, will not remove percolation.

\begin{figure}[H]
 \centering
 \includegraphics[scale=0.8]{oppg_a_percmatrix.png}
 \caption{A percolating cluster (light blue) on an $800\times800$ matrix with porosity $p=0.6$.}
 \label{perc}
\end{figure}

\subsection{Molecular dynamics}\label{MD}
Molecular dynamics is the simulation of the dynamics of atoms and molecules using classical, newtonian mechanics in the sense that the molecules are affected by a potential, and that the sum of forces describes the dynamics. 
Their dynamics are then integrated forward in time, and used to describe for example flow in nanoporous media. This means that the system is fully described by the position and velocities of all the atoms. 
Again, there is a vast varitety in the level of complexity here and we will only look to the simplest example, namlely the Lennard-Jones potential \ref{LJ}. 
This potential consists of an $r^{-12}$ term which denotes the Pauli repulsion at short ranges, and an $r^{-6}$ long range, attractive Van Der Waals term. $r$ denotes the distance between two atoms. 
The Lennard-Jones potential is derived to simulate Argon in the Van Der Waals equation of state.
\begin{equation}\label{LJ}
 U = 4\epsilon\left[\left(\frac{\sigma}{r}\right)^{12}-\left(\frac{\sigma}{r}\right)^{6}\right]
\end{equation}
It is possible to do simulations of flow in nanoporous materials using the Lennard-Jones potential, although it is a far from perfect model, by only allowing some of the atoms to move. 
We will then have a matrix of stationary atoms, simulating a wall (note that there will still be forces acting from these atoms), and a liquid inside this matrix.

\section{Outlook}
As should now be apparent, diffusion in the ECS is a complex problem which is not fully understood.
% Figure - subfigure template (2 by 2 subfigures)
% \begin{figure}[H]
% \centering
%   \begin{subfigure}[b]{0.48\textwidth}
%     \includegraphics[width=\textwidth]{filename}
%     \caption{small caption}
%     \label{KM:age}
%   \end{subfigure}
%   \begin{subfigure}[b]{0.48\textwidth}
%     \includegraphics[width=\textwidth]{filename.png}
%     \caption{small caption}
%     \label{KM:treat}
%   \end{subfigure}
%   
%   \begin{subfigure}[b]{0.48\textwidth}
%    \includegraphics[width=\textwidth]{filename.png}
%    \caption{small caption}
%    \label{KM:asc}
%   \end{subfigure}
%   \begin{subfigure}[b]{0.48\textwidth}
%    \includegraphics[width=\textwidth]{filename.png}
%    \caption{small caption}
%    \label{KM:sex}
%   \end{subfigure}
%   \caption{some caption}
%   \label{KM}
% \end{figure}

\printbibliography
\end{document}

