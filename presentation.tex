\documentclass{beamer}

\usepackage[utf8]{inputenc}
\usepackage{default}
\usepackage{graphicx}
\usepackage{subfig}
\usepackage{rotating}

\usepackage[]{biblatex}
% \bibliography{refs/refs.bib}

% \pdfminorversion=5

%-----------------------------------------------------%

\newcommand{\OP}[1]{{\bf\widehat{#1}}}
\newcommand{\ket}[1]{\left| #1 \right>}
\newcommand{\bra}[1]{\left< #1 \right|}
\newcommand{\braket}[2]{\left\langle #1 | #2\right\rangle}
\newcommand{\shift}{\vspace{0.5cm}\pause\\}
\newcommand{\qqq}{\qquad\qquad\qquad}
\newcommand{\qq}{\qquad\qquad}
\newcommand{\rot}[1]{\begin{sideways}#1\end{sideways}}
\newcommand{\OBDscale}{0.125}
\newcommand{\rvec}{\textbf{r}}
\newcommand{\alphavec}{\boldsymbol{\alpha}}
\renewcommand{\d}{\partial}
%-----------------------------------------------------%

\newcommand{\subfigure}{\subfloat}

\usetheme{Warsaw}
\title[Presentation - Computational neuroscience]{Diffusion processes in the extracellular space}
\author{Fredrik E. Pettersen}
\date{\today}



\begin{document}

\begin{frame}
\titlepage
\end{frame}



\begin{frame}
 \frametitle{Contents}
 \tableofcontents[hideallsubsections]
\end{frame}

\section{Introduction}
\begin{frame}
 \frametitle{Brief introduction to the brain}
 \begin{figure}
  \centering
  \includegraphics[scale=0.8]{presentation_brain.jpg}
 \end{figure}
\end{frame}

\begin{frame}
 \frametitle{Cells in the brain}
 \begin{columns}
\column{2.0in} Neurons:\\
\begin{itemize}
 \item Signal processing
\end{itemize}
\begin{figure}[H]
 \centering
 \includegraphics[width=\textwidth]{neuron.jpg}
\end{figure}

\column{2.0in} Neuroglia:\\
\begin{itemize}
 \item Janitorial tasks
\end{itemize}
\begin{figure}[H]
 \centering
 \includegraphics[width=\textwidth]{neuroglia.png}
\end{figure}
 \end{columns}
\end{frame}

\begin{frame}
 \frametitle{The extracellular space}
 \begin{columns}
 \column{2.0in}
 \begin{itemize}
  \item Space surrounding neurons and neuroglia
  \item Accounting for $\sim20\%$ of total brain volume
  \item Important for transport of nutrients, medicines etc.
 \end{itemize}

\column{2.0in}
 \begin{figure}[H]
  \includegraphics[width=\textwidth]{brain.jpg}
  \caption{Extracellular space marked as dark grey.}
 \end{figure}
 \end{columns}

\end{frame}

\section{Mathematical models}

\begin{frame}
 \frametitle{Basic diffusion}
 The basic diffusion equation reads
 \begin{equation}
  \frac{\d C}{\d t} = D\nabla^2C
 \end{equation}
Einstein relations
\begin{equation}
 D = \frac{k_B T}{6\pi\eta r}
\end{equation}
\begin{equation}
 \langle r^2\rangle = 2dDt
\end{equation}
\end{frame}

\begin{frame}
 \frametitle{Diffusion in ECS}
\begin{columns} 
 \column{2.0in} 
 Network simulations: \\
 \begin{itemize}
  \item Verification against experimental results
  \item Local field potential
  \item Extracellular conductance
 \end{itemize}
 \begin{equation}
  \sigma = \frac{cq}{k_B T}D 
 \end{equation}
 
\column{2.0in}
\begin{equation*}
\nabla\cdot(\sigma(\mathbf{r})\nabla\phi(\mathbf{r},t)) = -C(\mathbf{r},t)
\end{equation*}

\end{columns}

\end{frame}

\begin{frame}
 \frametitle{Modified diffusion equation}
 A modified version of the basic diffusion equation is needed to account for 
 \begin{itemize}
  \item Sources
  \item Uptake of diffusing molecules
  \item evt. bulk flow (absent below)
 \end{itemize}
This new equation reads
\begin{columns}
\column{2.0in}
\begin{equation}
   \frac{\d C}{\d t} = D^*\nabla^2C +\frac{s}{\alpha} -k'C
\end{equation}
\column{2.0in}
\begin{equation}
 \lambda = \sqrt{\frac{D}{D^*}}
\end{equation}

\end{columns}
 \end{frame}

 \begin{frame}
  \frametitle{Numerical simulations}
  This will be addressed in the Molecular Dynamics part.
 \end{frame}


\section{Measurement techniques}
 \begin{frame}
 \frametitle{Optical measurements}
 \begin{itemize}
  \item Developed in 1993 by Nicholson and Tao
  \item Based on thin slices and macromolecules with fluorescent label
  \item Problems with slice thickness and photobleaching
  \item Quantum dot IOI
 \end{itemize}
\end{frame}

 \begin{frame}
 \frametitle{TMA$^+$ measurements}
 \begin{itemize}
  \item Micro-pipette
  \item Ion sensitive micro-electrodes a known distance away ($\sim100\mu$m)
  \item Measures concentration curve at the electrode position
  \item Time resolution as well as spatial resolution
  \item Some refinements possible
 \end{itemize}
\end{frame}

 \begin{frame}
 \frametitle{Radiotracer measurements}
 \begin{itemize}
  \item ``Oldest trick in the book`` - most intuitive
  \item Good method, but slightly out dated due to radioactivity
  \item Results still used as verification
  \item Offers independent reference and cross-species data
 \end{itemize}

\end{frame}

 \begin{frame}
 \frametitle{Diffusion Tensor Imaging}
 \begin{itemize}
  \item Non invasive method based on MRI
  \item Immensely complicated; I have no idea what I'm doing!
 \end{itemize}

\end{frame}

\section{Other possible modeling approaches}
\begin{frame}
 \frametitle{Molecular dynamics}
 \begin{itemize}
  \item Study of systems of atoms and their time-development
  \item Most research towards fracture mechanics and flow in tight rocks
  \item Much of the geometry is similar, but the length scale is a bit to small
  \item Dissipative fluid dynamics
 \end{itemize}
 
\end{frame}

\begin{frame}
 \frametitle{My experiment - motivation}
 \begin{columns}
  \column{2.0in}
  \begin{itemize}
   \item Results from 2003 article by Hrab\v{e}tov\'{a} and Nicholson
   \item Max value of tortuosity $\lambda \leq 1.225$
   \item Diffusion modeling on regular geometries
  \end{itemize}
\column{2.0in}
\begin{figure}[H]
\centering
\includegraphics[width=\textwidth]{octahedra.png}
\end{figure}

 \end{columns}

\end{frame}

\begin{frame}
 \frametitle{My experiment - results}
 \begin{columns}
  \column{2.0in}
  \begin{itemize}
   \item Making spheres of stationary atoms
   \item Measuring self diffusion constant of liquid using Einstein relation
   \item Comparing to self diffusion constant of bulk fluid
   \item Found $\lambda \approx 1.41$
   \item Limitations
  \end{itemize}
\column{2.0in}
\begin{figure}[H]
\centering
\includegraphics[width=\textwidth]{nanoporous_fluid.png}
 \end{figure}

 \end{columns}

\end{frame}

\begin{frame}
 \frametitle{Random walks}
 \begin{columns}
  \column{2.0in}
  \begin{itemize}
   \item Percolation theory
   \item Random walks and diffusion
   \item Spanning cluster
   \item Results from by Hrab\v{e}tov\'{a} and Nicholson
   \item Limitations
  \end{itemize}
\column{2.0in}
\begin{figure}[H]
\centering
\includegraphics[width=\textwidth]{oppg_a_percmatrix.png}
 \end{figure}

 \end{columns}
\end{frame}

\section{Outlook}
\begin{frame}
 \frametitle{Outlook}
%   \begin{columns}
%   \column{2.0in}
  \begin{itemize}
   \item Unanswered questions (limitation of tortuosity)
   \item Other modeling methods
   \item Multi scale models; the best from both worlds?
  \end{itemize}
% \column{2.0in}
% \begin{figure}[H]
% \centering
% \includegraphics[width=\textwidth]{oppg_a_percmatrix.png}
%  \end{figure}
% 
%  \end{columns}
\end{frame}

\begin{frame}
% \printbibliography
\end{frame}
\end{document}